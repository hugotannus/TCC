\documentclass[a4paper,10pt]{article}

\usepackage[utf8x]{inputenc}
\usepackage[brazil]{babel}      %permite acentuação automática para o Portugues do Brasil,
\usepackage[T1]{fontenc}        %permite a hifenização correta em Portugues
%\usepackage[ansinew]{inputenc}  %permite a inserção de acentos (Windows)
\usepackage{amsfonts}
\usepackage[a4paper,top=2cm,bottom=2cm,left=2cm,right=2cm,nohead,nofoot]{geometry}
%\usepackage[lined,algo2e,ruled,portugues]{algorithm2e}
\usepackage{indentfirst}
\usepackage{lscape}
\usepackage{graphicx}

\newcommand{\fw}[1]{\emph{#1}}

%-------------------------------------------------------------------

\begin{document}

\author{Hugo Cabral Tannús \and Humberto José Longo}
\title{Proposta de Trabalho de Conclusão de Curso}
\date{\today}
\maketitle

\section{Introdu\c{c}\~{a}o}
\label{sec:intro}
Problemas de otimização combinatória são comumente utilizados em computação para
resolver problemas reais, geralmente, ligados à área de planejamento. Nesse
aspecto, são comuns buscar uma combinação de valores para dadas variáveis para
minimizar o custo financeiro de um projeto ou maximizar o lucro obtido com a
produção de quantidades diferenciadas de cada mercadoria. Não obstante, é comum
encontrar problemas em que as variáveis envolvidas apresentam a restrição de 
poderem assumir apenas valores inteiros; também é comum que o problema
apresente milhares de variáveis; e é possível, ainda que não haja um método
polinomial que resolva o problema, comprometendo, assim, as chances de construir
um algoritmo que consiga encontrar deterministicamente uma solução ótima, uma
vez que, para cada variável acrescentada ao problema, o custo para encontrar uma
solução ótima cresce exponencialmente. Tais problemas são classificados como
$\mathcal{NP}$-difíceis.

Nesse âmbito, o paradigma \fw{Branch \& Bound} (B\&B) é uma das principais 
ferramentas de construção de algoritmos que resolvam problemas de otimização
$\mathcal{NP}$-difíceis de maneira eficiente. Dado um determinado problema,
um algoritmo B\&B fará uma varredura em todo o espaço das possíveis soluções
(viáveis e inviáveis) em busca da melhor delas. Contudo, realizar simplesmente
uma enumeração de cada solução torna-se uma abordagem ineficiente, uma vez que
há um crescimento exponencial das soluções em potencial e, por isso, o algoritmo 
faz uso de
limites (``bounds'') para as funções a serem otimizadas, combinando os
resultados com o valor
da atual melhor solução. Essa abordagem se torna interessante em problemas de
larga escala, pois economiza recursos da máquina e viabiliza buscar uma solução
ótima, já que permite que o
algoritmo implicitamente realize a busca apenas em parte do espaço de soluções

% Em qualquer ponto durante o processo de solução, o status da solução em relação
% á busca do espaço de solução é descrita por um ``pool'' de subconjuntos ainda
% inexplorados desse espaço e a melhor solução encontrada até agora. Inicialmente,
% há apenas um único subconjunto, nominalmente o espaço de slução completo, e o
% melhor resultado é infinito. Subsepaços inexplorados são represtnados por nós
% em uma árvore de busca gerada dinamincamente que, inicialmente, contém apenas o
% nó raiz. Cada iteração de um algoritmo B\&B clássico processa um destes nós.
% A iteração tem três componentes principais: seleção do nó a ser processado,
% cálculo de limites (ou delimitação), e ramificação.
% 
% A sequência adotada para cada uma destas etapas variam de acordo com a etartégia
% escolhida para selecionar o próximo nó a ser porecessado. Por exemplo, se a
% seleção do próximo nó for baseada no valor limite dos subproblemas, então a
% primeira operação de uma iteração, após a escolha do nó, é a ramificação em dois
% ou mais nós (subespaços) que serão investigados em iterações subsequentes.
% Em cada ramificação realaizada, é verificado se o subespaço consiste em uma
% única solução, em cuso caso será comparado com a melhor solução atual e, então,
% mantida a melhor dentre elas; caso contrário, é calculado a função delimitadora
% para o subespaço e comparada com a melhor solução atual e, caso seja possível
% afirmar/estabelecer que o subespaço não pode conter uma solução ótima, então
% todo o sbespaço é descartado. Do contrário, ele é armazenado em um conjunto de
% nós viaeis juntamente com os seus limites. A busca termina quando não houverem
% mais partes inexploradas do espaço de solução, e asolução ótima será então o
% valor armazenado como sendo a ``atual melhor''.

A seção 2 trará um breve resumo sobre aspectos teóricos envolvidos no projeto.
A seção 2.1 irá definir o conceito de Programação Inteira, e as seções 2.1.1
tratará, especificamente, do problema da Mochila Multi-Dimensional. A seção 2.2
trarão comentários acerca  e o algoritmo \fw{Branch \& Bound}. A seção 3 destacará
algumas das principais
ferramentas computacionais para modelagem de problemas de programação inteira.
E, por fim, as seções 4 e 5 definirão a proposta para o Projeto Final do Curso,
bem como o cronograma das atividades a serem realizadas.

\section{Fundamentação Teórica}
\label{sec:teoria}

\subsection{Programação Inteira \& Inteira Mista}
\label{ssec:pi}
Denomina-se \fw{Programação Inteira} um problema de otimização que trabalha com
variáveis que podem assumir valores discretos (ou inteiros). Quando apenas parte
das variáves está sujeita a essa restrição, o problema passa a ser classificado
como \fw{Programação Inteira Mista}. Há ainda a \fw{Programação Binária} ou
\fw{Programação Zero-Um}, em que as variáveis estão restritas a receber apenas
dois valores (\fw{0} ou \fw{1}, \fw{verdadeiro} ou \fw{falso}, \fw{sim} ou
\fw{não}, etc).

\subsubsection{0--1 \fw{Multidimensional Knapsack Problem}}
\label{sssec:mdkp}
O problema das mochilas (0--1 \fw{Knapsack Problem}) consiste em, dado
um conjunto de objetos, cada um associado a um peso e um valor,
determinar quais deles devem ser escolhidos de modo que o peso total
seja menor que um dado limite e o valor total seja máximo. Embora este
problema pertença à classe dos problemas $\mathcal{NP}$-difíceis,
geralmente pode ser resolvido de forma eficiente com a utilização da
técnica de programação dinâmica nos casos em que os pesos não são
fracionários.

Considere uma mochila de capacidade $b$ e um conjunto de $n$ objetos,
identificados como $1,2,\dots,n$, cada um com peso de $a_j \in
\mathbb{Z}^+$ unidades e com valor $c_j$. Definindo-se que $x_j$ é uma
variável binária que indica se o objeto $j$ foi colocado na mochila
($x_j=1$) ou não ($x_j=0$), o problema é formulado como:

$$
 \rotatebox[origin=c]{90}{\small Knap} \,  \, \left\{
  \begin{array}{rrcl}
   \multicolumn{4}{l}{\mbox{Max} \quad z \> = \> c.x}    \\
   \multicolumn{4}{l}{\mbox{sujeito a}}                 \\
   & A.x & \leq & b            \\
   &   x & \in  & \{0,1\}^{n}  \\
  \end{array}
  \right.
$$

O problema das mochilas multidimensionais \fw{MdKP} (0--1
\fw{Multidimensional Knapsack Problem}) é uma generalização do
problema das mochilas definido anteriormente. Considere que dada uma
mochila de $m$ dimensões, cada uma de capacidade $b_i$, a matriz $A$,
de dimensão $m \times n$, representa o peso dos $n$ objetos para cada
uma das $m$ dimensões da mochila. O objetivo é encontrar um
subconjunto dos $n$ objetos cujo valor total seja máximo e cujo peso
total respeite o limite de cada uma das dimensões da mochila.
Novamente definindo-se que $x_j$ é uma variável binária para indicar a
escolha ou não do objeto $j$, então o problema é formulado como:

$$
 \rotatebox[origin=c]{90}{\small MdKP} \,  \, \left\{
  \begin{array}{rrclc}
   \multicolumn{4}{l}{\mbox{Max} \quad z \> = \> c.x}    \\
   \multicolumn{4}{l}{\mbox{sujeito a}}                 \\
   & A_{i}.x & \leq & b_{i}       & i \> = \> 1,\dots, m \\
   &       x & \in  & \{0,1\}^{n} &                      \\
\end{array}
\right.
$$
%opening

\subsection{\fw{Branch \& Bound}}
\label{ssec:b&b}
O algoritmo \fw{Branch \& Bound} (B\&B) é um algoritmo enumerativo, organizada em uma
estrutura de árvore onde cada nó representa uma espaço de soluções possíveis.
Cada ramificação na árvore corresponde a uma parte do espaço de soluções
representados pelo nó pai, por isso também denominada de subespaço. Essa estrutura
permite que todos as possíveis soluções de um problema de programação inteira
sejam enumeradas de maneira implícita ou explícita, garantindo assim que todas as
soluções ótimas sejam encontradas.

Cada iteração de um algoritmo B\&B clássico processa um destes nós. Uma iteração
é composta,
fundamentalmente, por três etapas: etapa de separação (ou ramificação),
relaxação e seleção do próximo nó (ou sondagem). Na etapa de separação, o
problema original $(P)$ é
separado em $q$ subproblemas $(P_1),(P_2),\dots,(P_q)$ sujeitos às
seguintes condições:

\begin{description}
 \item[(S1)] Toda a solução viável de $(P)$ é uma solução de somente um dos subproblemas $(P_i)$, $i=1,2,\dots,q$.
 \item[(S2)] Uma solução viável de qualquer um dos subproblemas $(P_i)$, $i=1,2,\dots,q$ é, também, uma solução viável de $(P)$.
\end{description}

Estas condições asseguram que o conjunto das soluções viáveis de cada um dos subproblemas $(Pi)$, $i = 1, 2,
\dots, q$ é uma partição do conjunto das soluções viáveis de $(P)$. Os subproblemas $(Pi)$, $i = 1, 2, \dots, q$ são
denominados descendentes de $(P)$ e podem, sucessivamente, gerar seus próprios descendentes.

A relaxação consiste em, temporariamente, ignorar algumas restrições do problema $(P)$ visando torná-lo
mais fácil de resolver. A condição que deve ser satisfeita é que o conjunto de soluções viáveis do problema
original $(P)$ esteja contido no conjunto de soluções viáveis do problema relaxado $(P_R)$. Isto implica que:

\begin{description}
  \item[(R1)] Se $(P_R)$ não tem solução viável, então o mesmo é verdadeiro para $(P)$.
  \item[(R2)] O valor mínimo de $(P)$ não é menor que o valor máximo de $(P_R)$.
  \item[(R3)] Se uma solução ótima de $(P_R)$ é viável em $(P)$, então ela é uma solução ótima de $(P)$. 
\end{description}

Dentre as formas possíveis de relaxação, destaca-se a eliminação das restrições de integralidade das
variáveis, o que transforma o problema inteiro misto em um problema linear (PL) padrão.
Na análise dos problemas candidatos, é necessário determinar quais são promissores e, portanto, devem ser
examinados, e quais podem ser sumariamente descartados. Isto é realizado na etapa de sondagem onde o
problema candidato $(PC)$ é eliminado (descartado para análises futuras), juntamente com todos os seus
descendentes, se satisfizer à pelo menos um dos seguintes critérios:

\begin{description}
  \item [(CS1)] O problema candidato relaxado $(PC_R)$ não tem solução viável. Devido a (R1), isto significa que
o problema candidato $(PC)$ também não tem solução viável.
  \item [(CS2)] A solução ótima do problema candidato relaxado $(PC_R)$ é pior (\fw{bounding}) do que a melhor
solução atualmente conhecida para $(P)$ (solução incumbente). Observar que a solução ótima do
problema candidato relaxado é sempre melhor ou igual à solução do problema candidato e de seus
descendentes.
  \item [(CS3)] Uma solução ótima do problema relaxado $(PC_R)$ é viável, também, em $(PC)$. Neste caso, devido
a (R3), ela é ótima em $(PC)$ e, devido a (S2) ela é também factível em $(P)$. Caso seja melhor que a
incumbente atual, a solução deste problema candidato passa a ser a nova incumbente. 
\end{description}

\section{\fw{MIP Solvers}}
\label{solvers}
A implementação de um algoritmo Bench \& Bound exige que seja definido um limite
superior ou inferior para o problema. Para tanto, utiliza-se de algumas
ferramentas de modelagem de programação inteira, conhecidas como \fw{Mixed
Integer Programming Solvers}. Para este projeto, será escolhido uma dentre
quatro destes solucionadores como ferramenta para a modelagem e implementação do
problema, sendo dois deles não comerciais e dois comerciais, que são,
respectivamente: \fw{GLPK}, \fw{SCIP}, \fw{Cplex} e \fw{Gurobi}.

GLPK - \fw{GNU Linear Programming Kit} - é um pacote de software destinado à
resolução de problemas de programação linear em larga escala e de programação
inteira mista. É um dos MIP Solvers mais utilizados, já que está disponível nos
repositórios de pacotes das principais distribuições linux existentes; em
contrapartida ao desempenho e eficácia inferiores se comparado a outros
solucionadores. Entre os principais componentes inclusos no GLPK, estão:
os métodos simplex primal e dual e o método \fw{Branch \& Cut}; uma API para a
linguagem C, e; o suporte para a linguagem de modelagem \fw{GNU MathProg}.

O SCIP - \fw{Solving Constraint Integer Programs} - desenvolvido e mantido pelo \fw{Zuze Institute Berlin} (ZIB), está atualmente entre os MIB Solvers não comerciais de melhor desempenho. Voltado para programação inteira e para a abordagem \fw{Branch, Cut \& Price}, o SCIP é um framework flexível, uma vez que permite ao usuário adicionar funcionalidades pro meio de plugins, além de dar suporte a outros solucionadores de programação linear, que incluem, inclusive, os solvers CPLEX e o GUROBI. Disponível para sistemas Windows, Linux, MacOS e SunOS.

O CPLEX é um solucionador de problemas matemáticos de alto desempenho desenvolvido pela IBM, destinado a resolver problemas de programação linear, programação inteira mista e programação quadrática. É uma ferramenta estável, além de contar com o suporte técnico da IBM, o que a torna uma ferramenta confiável. Possui ambiente de desenvolvimento integrado e linguagem de modelagem descritiva, além de disponibilizar várias opções de algoritmos e de modelos de entrada e saída de dados. O CPLEX está disponível para diferentes sistemas operacionais, que incluem o Windows, o Linux e o MacOS.

Por fim, o Gurobi Optimizer, assim como o CPLEX, é voltado para programação linear, programação inteira mista e programação quadrática, diferenciando-se da ferramenta anterior pelo fato de ter sido projetada para trabalhar com programação paralela, por meio do uso de múltiplos processadores. Inclui implementações de alto desempenho para os métodos Simplex primal e dual, além de métodos como de ``corte de planos'' e heurísticas para modelos MIP e interfaces orientadas a objeto para as linguagens C++, Java, Python e .Net. Possui versões para as plataformas Windows, Linux, MacOS, dentre outros, além de oferecer licenças gratuitas para estudantes acadêmicos.

\section{Proposta}
\label{sec:prop}

Resolver um problema de programação inteira, mais especificamente o problema da
Mochila Multi-dimensional, utilizando o algoritmo de \fw{Branch \& Bound}, por 
meio de uma ferramenta de resolução de problemas inteiros Mistos.

\section{Cronograma}
\label{sec:cron}

\begin{table}[h]
       \centering
       \begin{tabular}{|l|p{.7cm}|p{.7cm}|p{.7cm}|p{.7cm}|p{.7cm}|p{.7cm}|p{.7cm}|p{.7cm}|}
               \cline{2-9}
               \multicolumn{1}{c|}{} & \multicolumn{4}{|c|}{\bf PFC--1} &
\multicolumn{4}{|c|}{\bf PFC--2}\\
               \hline \textbf{Etapa} & Mar & Abr & Mai & Jun & Ago & Set & Out & Nov \\
               \hline Revisão bibliográfica & X & X & & & & & & \\
               \hline Estudo \& pesquisa: PL \fw{vs} PI & & X & X & & & & & \\
	       \hline Estudo \& pesquisa: \fw{MDKP} & & & X & X & & & & \\
	       \hline Estudo \& pesquisa: \fw{B\&B} & & & X & X & X & & & \\
               \hline Projeto algoritmico & & & X & X & & & &\\
               \hline Estudo \& pesquisa: heurísticas & & & & & X & X & X & \\
               \hline Implementação do algoritmo & & & & & X & X & & \\
	       \hline Definição de heurísticas & & & & & & X & X & \\
	       \hline Implementação de heurísticas & & & & & & & X & X \\
               \hline Testes e Comparação de Resultados & & & & & & & & X \\
	       \cline{1-9}
       \end{tabular}
       \caption{Cronograma a ser seguido durante o ano.}
\end{table}

\end{document}
